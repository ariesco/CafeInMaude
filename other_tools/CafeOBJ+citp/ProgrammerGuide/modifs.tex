%!TEX root = pg.tex

We present here the modifications that must be performed into the tool to support
CafeOBJ specifications, as well as to add the new commands defined in the previous
sections. We present here the modifications for the Constructor-based Inductive
Theorem Prover, but they are general and can be applied to any other tool.
%
Moreover, note that this modifications are performed \emph{in the interface},
while the tool itself remains unchanged.

First, we need to distinguish the language we are working with. Hence, we require:
\begin{itemize}
\item
A sort \verb"Language", that will be used by all the languages.

\item
Constants of sort \verb"Language" for each language. In our case we have defined,
in the module in charge of handling the database (\verb"THM-DATABASE-HANDLING")
the following:

{\codesize
\begin{verbatim}
  sort Language .
  ops maude cafeobj : -> Language [ctor] .
\end{verbatim}
}

\item
An attribute to store the selected language. We have added, also in
\texttt{THM-DATABASE-HANDLING}, the following attribute:

{\codesize
\begin{verbatim}
  op language`:_ : Language -> Attribute [ctor] .
\end{verbatim}
}

\item
Commands to deal with this attribute. In this case, we have to add two new
commands into the syntax (in \texttt{PROVE-COMMANDS}):

{\codesize
\begin{verbatim}
  op maude`language`. : -> @Command@ .
  op cafeOBJ`language`. : -> @Command@ .
\end{verbatim}
}

\item
Rules for dealing with these commands. In our case we require the following two
rules to be added to \texttt{THM-DATABASE-HANDLING}:

{\codesize
\begin{verbatim}
  rl [maude-specs] :
     < O : X@Database | input : ('maude`language`..@Command@), output : nil ,
                        language : L, Atts >
  => < O : X@Database | input : nilTermList, output : ('\n '\b 'Maude 'selected
                                                      'as 'current 'specification
                                                      'language. '\o '\n),
                        language : maude, Atts > .

  rl [cafe-specs] :
     < O : X@Database | input : ('cafeOBJ`language`..@Command@), output : nil ,
                        language : L, Atts >
  => < O : X@Database | input : nilTermList, output : ('\n '\b 'CafeOBJ 'selected 
                                                      'as 'current 'specification
                                                      'language. '\o '\n),
                        language : cafeobj, Atts > .
\end{verbatim}
}

\item
Every rule in charge of commands must use the \verb"maude" value. Note that the
rules in Section~\ref{subsec:ext} used the \verb"cafeobj" value.

\item
The attributes must be initialized when creating the initial system with a default
value (in general the language must be \verb"maude", since it is the original
language). We can just add the constant \verb"initCafeAttS" defined in the previous
section to the initial state:

{\codesize
\begin{verbatim}
  rl [init] : init 
  => [nil, < o : CITPDatabase | db : initialDatabase, input : nilTermList,
                                output : nil, default : 'CONVERSION,
                                pTree : null, currentGoal : 'nil,
                                showMod : false, tactic : 0,
                                tacticRec : ('SI 'CA 'CS 'TC 'IP),
                                language : maude, initCafeAttS >,
     ('\s '\s '\s '\s '\s '\b string2qidList(thm-banner) '\o '\n help-list)] .
\end{verbatim}
}

\end{itemize}

Notice that these indications are quite general and most of them are only required
the first time a new language is integrated.

Second, we have to merge our syntax with the tool syntax. This is achieved by:
\begin{itemize}
\item
Adding the new syntax if any command is modified. In our case the \verb"goal"
command has extra syntax, since it supports transitions, so we add: 

{\codesize
\begin{verbatim}
  op trans_=>_; : @Bubble@ @Bubble@ -> @SentenceSet@ .
  op ctrans_=>_if_; : @Bubble@ @Bubble@ @Bubble@ -> @SentenceSet@ .
  op trns_=>_; : @Bubble@ @Bubble@ -> @SentenceSet@ .
  op ctrns_=>_if_; : @Bubble@ @Bubble@ @Bubble@ -> @SentenceSet@ .
\end{verbatim}
}

\item
Merging the modules containing the syntax for each tool. Since the syntax is
stored into the metarepresented module \texttt{thm-Grammar}, we can use the
following equation:

{\codesize
\begin{verbatim}
  eq thm-Grammar = addImports((including 'PROVE-COMMANDS .), CafeGRAMMAR) .
\end{verbatim}
}

In this way \verb"thm-Grammar" contains Full Maude syntax (that was already contained
in \texttt{CafeGRAMMAR}), the syntax to support CafeOBJ specifications and the
syntax for the CITP.
\end{itemize}

Third, we have to combine the behavior of both modules, so all the rules can be applied.
Since the rules dealing with CITP commands are executed by an object of class
\verb"CITPDatabaseClass", a subsort of \texttt{DatabaseClass}, we can add a new
subsort to merge both databases:

{\codesize
\begin{verbatim}
  subsort CITPDatabaseClass < CafeDatabaseClass .
\end{verbatim}
}

After applying this modifications, the languages are integrated and both of them can
be used independently.




