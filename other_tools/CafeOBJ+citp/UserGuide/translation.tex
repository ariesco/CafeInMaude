%!TEX root = ug.tex

We present here the details and limitations of the translation from CafeOBJ specifications
to Maude modules performed to introduce the former into the Full Maude database.

\subsection{Built-in modules}

CafeOBJ predefined modules are automatically mapped to the predefined Maude modules with
the same name. As said in the introduction, one of the benefits of CafeOBJ and Maude
being sister languages is that their share part of their syntax. This is quite convenient
in this case, since it allows us to reuse the Maude code. In other case, specific modules
defining the behavior of CafeOBJ modules should be used.

Note that some tools, as the Maude Formal Environment~\cite{mfe11}, does not allow the use
of predefined modules, so the user cannot import them if he want to use those tools.

%\subsection{Structured specifications\label{subsec:flat}}
%
%An important characteristic of this translation is that it considers \emph{flat specifications}, that is,
%every CafeOBJ specification is managed by the MFE as a set of equations and rules. This allows
%us to add some flexibility to the importations and the relations between modules and theories, as
%we will see in the next sections.

\subsection{Types of modules}

CafeOBJ allows the definition of two different kinds of module: modules with tight semantics,
introduced by the keyword \texttt{mod!}, and modules with loose semantics, introduced by
\texttt{mod*}. We translate the former to Maude modules (which will be functional modules
if not rules are specified, and system modules otherwise) and the latter to Maude theories (which
will be again translated as functional or system theories depending on the existence of rules).
%
CafeOBJ views will be always translated as Maude views.

\subsection{Parameterized modules and views}

As explained above, modules defined with syntax \verb"mod*"
have loose semantics and are thus translated as Maude theories. Since the current
version of Maude does not support parameterized theories, one of the main
limitations of the translation is that \emph{modules with loose semantics cannot
be parameterized}.
%
Regarding views, CafeOBJ syntax is richer than the one in Maude:
\begin{itemize}
\item
Views can be declared ``on the fly'' when instantiating a parameterized module.

\item
The views assigned to each parameter during the instantiation can be identified by the
parameter name, instead of introducing the views in the order given by the parameterized
module.

\item
Moreover, in CafeOBJ the syntax \verb"SortId.ParamId" is used for qualifying sorts, while in Maude
the syntax \verb"ParamId$SortId" is used.

\end{itemize}

Although the translation of each of these features is intuitive, is worth presenting a 
a small example. We start by translating the module \verb"M1", which requires
the existence of a sort \verb"A":

{\codesize
\begin{verbatim}
 (mod* M1 {
   [A]
 })
\end{verbatim}
}

This module is translated into a Maude theory as follows:

{\codesize
\begin{verbatim}
 (fth M1 is
   sort A .
 endfth)
\end{verbatim}
}

Now, we define a parameterized CafeOBJ module \verb"M2" with tight semantics, that receives
two parameters fulfilling the theory \verb"M1", \verb"X" and \verb"Y", and states that the sort
\verb"A" from \verb"X" is a subsort of a new sort \verb"B" declared in this module:

{\codesize
\begin{verbatim}
 (mod! M2(X :: M1, Y :: M1){
   [A.X < B]
 })
\end{verbatim}
}

This module is translated into Maude as follows:

{\codesize
\begin{verbatim}
 (fmod M2{X :: M1, Y :: M1} is
   sort B .
   subsort X$A < B .
 endfm)
\end{verbatim}
}

Now we define another module with tight semantics \verb"M3" that will be used to declare a
view \verb"V" from \verb"M1", mapping the sort \verb"A" from \verb"M1" to the sort
\verb"C" declared in \verb"M3":

{\codesize
\begin{verbatim}
 (mod! M3 {
   [C]
 })

 (view V from M1 to M3 {
   sort A -> C
 })
\end{verbatim}
}

These modules are translated into Maude in a straightforward way:

{\codesize
\begin{verbatim}
 (fmod M3 is
   sort C .
 endfm)

 (view V from M1 to M3 is
   sort A to C .
 endv)
\end{verbatim}
}

It is important to note that
the target module in a view \emph{cannot be a theory}. This is due to the fact that in Maude
this kind of view is still not instantiated and hence it requires the module where the instantiation
takes place to be also parameterized and use some of these parameters in the instantiation,
which is not required in CafeOBJ and would lead to illegal specifications.
%
We explain at the end of the section a way to overcome this problem.

Finally, we define a module \verb"M4" that instantiates \verb"M2" by using the view \verb"V" with
the parameter \verb"Y" and an on-the-fly view to instantiate \verb"X":

{\codesize
\begin{verbatim}
 (mod! M4 {
   pr(M2(Y <= V, X <= view to M3 {sort A -> C}))
 })
\end{verbatim}
}

The translation into Maude first requires the creation of a new view \verb"OTF-VIEW0",
where the prefix \verb"OTF" stands for \emph{on the fly} and the index is generated for each
new view. Once this view has been introduced in the Maude database, the views in the
instantiation are sorted to match the parameters in \verb"M2", and then the importation is
accepted:

{\codesize
\begin{verbatim}
 (view OTF-VIEW0 is
   sort A to C .
 endv)

 (fmod M4 is
   pr M1{OTF-VIEW0, V} .
 endfm)
\end{verbatim}
}

It is worth noting that, 
to allow a more flexible translation, we introduce an equivalent Maude module for each
theory introduced into the database. If it is required to use this module instead of the theory
to obtain a valid Maude specification, the system will use it and show a warning message to the
user. This feature can be turned off with the command:

{\codesize
\begin{verbatim}
 Maude> (strict translation on .)
 The modules will be introduced as originally written.
\end{verbatim}
}

\noindent
and used again by typing:

{\codesize
\begin{verbatim}
 Maude> (strict translation off .)
 The translation will adapt CafeOBJ specifications to meet Maude requirements when 
possible.
\end{verbatim}
}

\subsection{Importations}

Maude importation features present some limitations:

\begin{itemize}
\item
Theories can only be imported by other theories, and only in \verb"including" mode.

\item
Modules can be imported by theories. Note that this importation indicates that, if the
theory is used as the source of a view the target module must also import the module.

\item
Modules cannot import theories, only modules.
\end{itemize}

Some of the specifications thus
obtained are not valid Maude specifications from the executability point of view, and hence
commands like \texttt{red}, \texttt{rew}, or \texttt{search} cannot be used on them.
%
Under these conditions, the importations allowed by our translation are, in addition to all the
already available in Maude:
\begin{itemize}
\item
We allow the importation of theories by other theories in any mode, since they will be
just introduced into the theory assuming \verb"including" mode.

\item
Modules can be imported by theories, but the conditions presented above must hold.

\item
importing a \verb"mod*" within a \verb"mod!" is allowed, and the tools will use the equations
and rules in these theories in the standard way.
\end{itemize}

As explained above, we can use the commands \verb"(strict translation on .)" and
\texttt{(strict translation off .)} to force the system to accept the exact CafeOBJ
specification.

%Also note that equations (respec.\ rules) that are not expected to be used by the MFE can
%be declared with the \texttt{nonexec} attribute, that indicates that the equations (respec.\ rules)
%are not expected to be executed.

\subsection{Operators}

Both CafeOBJ operators and behavioral operators are straightforwardly mapped to
Maude operators. In the case of the CafeOBJ operator attributes \verb"l-assoc"
and \verb"r-assoc" attributes, which are not available in Maude, they are
translated by using \verb"gather".

\subsection{Equations}

CafeOBJ equations are directly translated into Maude equations.

\subsection{Rules}

\verb"trans" and \verb"btrans" (as well as \verb"ctrans" and \verb"cbtrans") declarations are
straightforwardly translated into Maude rules.

