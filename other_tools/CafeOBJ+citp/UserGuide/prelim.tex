%!TEX root = ug.tex

We present in this section the main features of CafeOBJ and Maude.

\subsection{CafeOBJ\label{subsec:cafe}}

CafeOBJ~\cite{cafe-report}

As an example of CafeOBJ specification we will use a modified version of the example presented
in~\cite{mfe11}. We first describe natural numbers in the
\texttt{MNAT} module. This module is defined with syntax \texttt{mod!}, which indicates that
the module has tight semantics, and includes the \texttt{TRUTH-VALUE} module in protecting
mode, denoting that no junk and no confusion can be added to the predefined sort \texttt{Bool}
defined there:

\subsection{Maude\label{subsec:maude}}

%Maude~\cite{maude-book} is a high-level language and high-performance system,
%supporting both equational and rewriting logic computation.
Maude modules are executable rewriting logic specifications.
Rewriting logic~\cite{Meseguer92-tcs} is a logic of change very suitable
for the specification of concurrent systems that is parameterized by an
underlying equational logic, for which Maude uses membership
equational logic (\emph{MEL})~\cite{BouhoulaJouannaudMeseguer00},
which, in addition to
equations, allows one to state membership axioms characterizing
the elements of a sort. Rewriting logic extends \emph{MEL} by adding rewrite rules.

Maude modules are executable
rewriting logic specifications.
%
Maude functional modules \cite[Chap.~4]{maude-book}, introduced
with syntax \texttt{fmod ...\ endfm}, are executable membership
equational specifications 
that allow the definition of
sorts (by means of keyword
\texttt{sort}(\texttt{s})); subsort relations between sorts
(\texttt{subsort}); operators (\texttt{op}) for building values of these
sorts, giving the sorts of their arguments and result, and which may have
attributes such as being associative (\texttt{assoc}) or commutative
(\texttt{comm}), for example; memberships (\texttt{mb}) asserting that a term
has a sort; and equations (\texttt{eq}) identifying terms.
Both memberships and equations can be conditional (\texttt{cmb} and \texttt{ceq}).
%
Maude system modules \cite[Chap.~6]{maude-book}, introduced with syntax \texttt{mod ...\ endm},
are executable rewrite theories. A system module can contain all the
declarations of a functional module and, in addition, declarations for rules (\texttt{rl})
and conditional rules (\texttt{crl}).


Full Maude is an extension of Maude written in Maude itself~\cite[Part II]{maude-book}. Besides
providing an even more powerful module algebra than the one available in Core Maude,
Full Maude has been traditionally used as a basis for further extensions. It is possible
to change the syntax of existing features and add new kinds of modules and commands.
Moreover, it is possible to build interactive tools because Full Maude is built on top
of the Loop Mode~\cite[Chapter~17]{maude-book}, which provides a mechanism to parse the
modules and commands introduced by the user \emph{enclosed in parentheses}. The specific
features required by Full Maude syntax are shown in Appendix~\ref{app:java}.


