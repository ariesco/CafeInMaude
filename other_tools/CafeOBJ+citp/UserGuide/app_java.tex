%!TEX root = ug.tex

Hence, in order to make CafeOBJ specifications~\cite{cafe-report} compatible with Full Maude it
is necessary to perform some modifications. Namely, we have to:
\begin{itemize}
\item
Enclose every module into parentheses.

\item
Introduce the symbol \verb"`" before the scape characters, such as parentheses, commas,
or curly braces, when defining operators.

\item
Transform CafeOBJ comments into Maude comments.
\end{itemize}

Moreover, we consider that the \texttt{metadata} attribute might be quite useful in some cases, and it
is used by several Maude tools, like the prover in Section~\ref{sec:using:citp}. Thus, it
is possible to write metadata information following equations and transitions as CafeOBJ
comments (hence being valid CafeOBJ specifications). These comments must be introduced later
into the statements to work.

We have implemented a Java application that, given a CafeOBJ specification (possibly with
commented metadata annotations), generates a new file performing the modifications explained
above. In this way, if we have a file \verb"test.cafe" with the following CafeOBJ specification

{\codesize
\begin{verbatim}
mod! TEST {
 [Foo]
 op a : -> Foo
 op b : -> Foo
 op {_,_} : Foo Foo -> Bool {comm}
 --> We just check whether the values are equal
 eq [l1] : {a, a} = true .
 eq [l2] : {b, b} = true .
 eq [l3] : {a, b} = false . **> {metadata "different"}
}
\end{verbatim}
}

We can download the application \verb"parser.jar", available at \url{http://maude.sip.ucm.es/cafe/},
and use the command

{\codesize
\begin{verbatim}
$ java -jar parser.jar test.cafe test-trans.cafe
\end{verbatim}
}

\noindent
where \verb"test-trans.cafe" is the name of the target file to be created, we obtain the following
transformed module:

{\codesize
\begin{verbatim}
(mod! TEST {
 [Foo]
 op a : -> Foo
 op b : -> Foo
 op `{_`,_`} : Foo Foo -> Bool {comm}
 --- We just check whether the values are equal
 eq [l1] : {a, a} = true .
 eq [l2] : {b, b} = true .
 eq [l3] : {a, b} = false {metadata "different"} .
}
)
\end{verbatim}
}

Note that the module is now enclosed in parentheses, the scape characters in the third
operator declaration are now preceded by \verb"`", while the rest of scape characters
remain unchanged, the comment has been transformed into a Maude comment and the metadata
attribute has been introduced into the equation.

Once the files we want to load have been transformed with this application, we have to load
Full Maude and \verb"cafeOBJ2maude.maude", available at \url{http://maude.sip.ucm.es/cafe/}:

{\codesize
\begin{verbatim}
Maude> loop cafe2maude-init .

     CafeOBJ2Maude 1.0 started.
 CafeOBJ specifications can be introduced now into the Full Maude database.
\end{verbatim}
}

This command starts an input/output loop where the modules previously obtained can be
introduced. Now we can use the standard Maude commands to load modules:

{\codesize
\begin{verbatim}
Maude> in test-trans.cafe
Introduced Tight Cafe Module: TEST
\end{verbatim}
}

