%!TEX root = ug.tex

We describe in this section how to use our tool to introduce CafeOBJ specifications
into the Full Maude database. We will use this tool as base to develop the tools in
the next sections.

\subsection{Configuring the tool\label{subsec:c2m_config}}

The first step for configuring the tool is downloading the \verb"parser.jar",
\verb"cafeOBJ2maude.maude", and \verb"cafe2maude" files, available at
\url{http://maude.sip.ucm.es/cafe/}. Then, the user must edit
(with any text editor) the \verb"cafe2maude" file by fixing the values of the variables:
\begin{description}
\item[\texttt{MAUDE},] which keeps the path to the Maude binary (available at
\url{http://maude.cs.uiuc.edu/}).

\item[\texttt{CAFE2MAUDE},] which keeps the path of to the \verb"cafeOBJ2maude.maude"
file described above.

\item[\texttt{PARSER},] which keeps the path of to the \verb"parser.jar"
file described above.
\end{description}

Moreover, the file \verb"full-maude.maude" must be available either in the same directory
as Maude or in the current directory.

\subsection{Using the tool\label{subsec:c2m_using}}

To start the tool it is enough to execute the \verb"cafe2maude" script configured in the
previous section by:

{\codesize
\begin{verbatim}
$> ./cafe2maude file_1 ... file_n
\end{verbatim}
}

\noindent
where \verb"file_1 ... file_n" refer to the CafeOBJ files we want to introduce.
The \verb"cafe2maude" script modifies the CafeOBJ files by using \verb""parser.jar"
to add the features required by Full Maude, and then creates a temporal file with
these modified files. This file is the one loaded by Full Maude. 
%
The particularities of this translation can be found in Appendix~\ref{sec:trans},
while Appendix~\ref{app:java} describes the transformations performed by the Java file,
and how to use it in an isolated way.

The modules introduced in this way will behave as standard Maude modules, and hence
they can:
\begin{itemize}
\item
Be imported, that is, they can be imported by other CafeOBJ and \emph{Maude} specifications,
for example we can introduce the Maude module:

{\codesize
\begin{verbatim}
(fmod EXT-TEST is
  pr TEST .
  
  sort SFoo .
  subsort Foo < SFoo .
endfm)
\end{verbatim}
}

However, the importations are restricted to those allowed by Maude. See
Section~\ref{sec:trans} for details.

\item
Be used as source or target module in a view. Any combination with Maude
theories or modules is allowed. For example, we can define the following
view using the \verb"TEST" module:

{\codesize
\begin{verbatim}
view Test from TRIV to TEST is
  sort Elt to Foo .
endv
\end{verbatim}
}

\item
Import Maude modules, that is, CafeOBJ specifications can import Maude
modules. For example, we can 

{\codesize
\begin{verbatim}
mod! TEST2 {
  pr(EXT-TEST)
  [Value]
}
\end{verbatim}
}

\item
Show the original CafeOBJ modules. We can use the command
\texttt{(original CafeOBJ module MODULE-NAME .)} to print the CafeOBJ specification
initially introduced into Full Maude. For example, the original \verb"TEST" module
is shown as follows:

{\codesize
\begin{verbatim}
Maude> (original CafeOBJ module TEST .)

mod! TEST{
  [Foo]
  op a : -> Foo {constr}
  op b : -> Foo {constr}
  op `{_`,_`} : Foo Foo -> Bool {comm}
  eq [l1] : {a,a} = true .
  eq [l2] : {b,b} = true .
  eq [l3] : {a,b} = false {metadata "different"} .
}
\end{verbatim}
}

\item
Show the transformed modules. The Maude command \verb"(show module MODULE-NAME .)"
allows the user to visualize all the modules thus far introduced. For example, the
following command is used to show the module \verb"TEST2":

{\codesize
\begin{verbatim}
Maude> (show module TEST2 .)
fmod TEST2 is
  including BOOL .
  protecting EXT-TEST .
  sorts Value .
endfm
\end{verbatim}
}

\item
Show the views. Similarly to the command above, the command \verb"(show view VIEW-NAME .)"
allows the user to print the views. For example, we can use it to show the \verb"Test"
view as follows:

{\codesize
\begin{verbatim}
Maude> (show view Test .)
view Test from TRIV to TEST is
  sort Elt to Foo . 
endv
\end{verbatim}
}

\item
Use all the evaluation Maude commands, including \verb"red", \verb"rew", and \verb"search".
For example, we can reduce a term in \verb"TEST" as follows:

{\codesize
\begin{verbatim}
Maude> (red in TEST : {a, b} .)
reduce in TEST : {a,b}
result Bool : false
\end{verbatim}
}

Note that, since the \verb"red" command is available and importing Maude modules is possible,
we can also use the Maude model checker~\cite[Chapter 13]{maude-book}. 

\end{itemize}






