%!TEX root = ug.tex

We briefly present in this section how to use the Maude Formal Environment (MFE) with
CafeOBJ specifications. An exhaustive description of the system is available in~\cite{mfe11}.

First, it is important to note that some of the tools in the MFE do not support built-in operators,
so it cannot prove properties on specifications containing this kind of functions.
%
Hence, programmers cannot import modules using built-in functions into their specifications,
which includes standard modules such as \texttt{BOOL} or \texttt{NAT}.
%
In the \texttt{BOOL} case, the user is encouraged to use the \texttt{TRUTH-VALUE} module,
which defines the sort \texttt{Bool} and the constants \texttt{true} and \texttt{false}. This module
allows us to use conditional equations and rules, that require Boolean values in the conditions.
The rest of operations, including equalities, must be defined manually by the user.

The MFE must be downloaded and installed following the instructions
in~\url{http://maude.lcc.uma.es/MFE/}. Then the \texttt{cafeOBJ2maude.maude} file must
be added to the main folder (called \texttt{mfe}), while the \texttt{mfe.maude} file must be
replaced by its extended version supporting CafeOBJ specifications.
These files are available at \url{http://maude.sip.ucm.es/cafe}.

Once the tool is installed, it is started by loading the \verb"mfe.maude" file into the Maude system:

{\codesize
\begin{verbatim}
$ maude 
		     \||||||||||||||||||/
		   --- Welcome to Maude ---
		     /||||||||||||||||||\
	    Maude 2.6 built: Dec 10 2010 11:12:39
	    Copyright 1997-2010 SRI International
		   Thu Aug  9 13:41:39 2012

Maude> load mfe.maude
	    Full Maude 2.6.1d July 7th 2012

The Maude Formal Environment 1.0
    Inductive Theorem Prover - July 20th 2010
    Sufficient Completeness Checker 2a - August 2010
    Church-Rosser Checker 3m - July 7th 2012
    Coherence Checker 3l - November 24th 2010
    Maude Termination Tool 1.5i - July 7th 2012
\end{verbatim}
}

Once the MFE is loaded we can directly load our CafeOBJ specifications by enclosing each module or
view in parenthesis. If we introduce the appropriate parenthesis in the example presented in
Section~\ref{subsec:cafe}, assuming it is saved in a file called \texttt{example.cafe}, it is loaded as
follows:

{\codesize
\begin{verbatim}
Maude> load example.cafe
\end{verbatim}
}

It is important to note that this tool will use \emph{flat} modules, so the specifics
semantics required by the importations will not be taken into account.
This follows the approach used for Maude specifications, where these constraints are also
skipped. This is possible because the tools work on \emph{sets} of axioms, equations, and
rewrite rules.

We can use now all the tools in the MFE to analyze our CafeOBJ specifications.
We can start by checking
















