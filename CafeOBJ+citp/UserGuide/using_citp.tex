%!TEX root = ug.tex

We present in this section how to use the Constructor-based Inductive Theorem Prover
(CITP) with CafeOBJ specifications. A comprehensive explanation of the CITP can be
found in \url{http://www.jaist.ac.jp/~danielmg/citp.html}.

The CITP is started by configuring and executing the \verb"citp" script available
\url{http://maude.sip.ucm.es/cafe/}, followed by the CafeOBJ modules to be used,
as shown in the previous sections.
Note that this script automatically applies the transformation described in Section~\ref{sec:intro},
so the actual CafeOBJ files being used can be introduced.
%It starts an input-output loop where Maude modules and commands can be
%introduced by enclosing them in parentheses, as shown in Section~\ref{sec:intro}.
%Moreover, the \verb"ui.maude" file automatically loads the \verb"cafeOBJ2maude.maude" file,
%so CafeOBJ specifications can also be introduced.
%
Although we could work with the transformed Maude
versions obtained from the introduced modules, it is more convenient to allow CafeOBJ users to
work with the tool in a transparent way. To allow that, the tool has been extended with two commands:
\begin{itemize}
\item
\verb"(cafeOBJ language .)", which informs the tool that we want to work with CafeOBJ
specifications, and hence no mentions to Maude code will be done.

\item
\verb"(maude language .)", which informs the tool that we want to forget about CafeOBJ
syntax and use Maude syntax for commands and visualization.
\end{itemize}

By using the \verb"(cafeOBJ language .)" option a number of commands in the CITP are
modified from the interface point of view. We will consider in the following that
this command has been used:

{\codesize
\begin{verbatim}
Maude> (cafeOBJ language .)
CafeOBJ selected as current specification language.
\end{verbatim}
}

The revised commands are:

\begin{itemize}
\item
\verb"(goal ModuleName |- Equations/Rules/Memberships)", which has been adapted to use
CafeOBJ syntax and thus only accepts (possibly conditional) equations and transitions.
Moreover, we can also use the standard syntactic sugar from CafeOBJ to avoid writing
the variable sort once it has already been stated.
For example, we can define a simple module \verb"SPEC" (already modified by using the
application shown in Section~\ref{sec:intro}) for integers with the \verb"0" constant,
the successor and predecessor constructors, equations for representing the behavior of
these operators combined,
and a rule that decreases terms expressed by using successor:

{\codesize
\begin{verbatim}
(mod! SPEC {
  [Zero < Int]
  op 0 : -> Zero {constr}
  ops s_ p_ : Int -> Int {constr}
  eq [eq1] : s p X:Int = X .
  eq [eq2] : p s X:Int = X .
  trans [tr1] : s(0) => 0 .
})
\end{verbatim}
}

And now introduce a simple goal with two sentences. The first one
will be an equation stating that the successor of any number is equal to
applying the predecessor function once and the successor function twice to
that number. The second one is a transition which indicates that we reach
the integer \verb"X" if we start from a term applying the successor function
twice to it:

{\codesize
\begin{verbatim}
Maude> (goal SPEC |- eq s(X:Int) = p(s(s(X))) ; trans s(s(X:Int)) => X ;)
============================ GOAL 1-1 ============================
< Module SPEC is concealed
...
end
,
	  eq s X:Int
	    = p s s X:Int .
	  trans s s X:Int
	    => X . >
unproved

INFO: an initial goal generated!
\end{verbatim}
}

It is important to note that, since Maude does not allow importation of theories
in protecting mode, this tool considers that the intended semantics is that of
protecting. This is coherent with the approach presented in the previous section,
where the importation modes were modified to including.

\item
\verb"(set module on .)", which asks the tool to show the current module in every
proof step of the proof. If we activate this mode and then apply a step requiring
the module to be shown, it will be displayed by using CafeOBJ syntax. For example,
we can activate this mode

{\codesize
\begin{verbatim}
Maude> (set module on .)
INFO: Module will be displayed in goals
\end{verbatim}
}

\noindent
and introduce the goal shown above:

{\codesize
\begin{verbatim}
Maude> (goal SPEC |- eq s(X:Int) = p(s(s(X))) ; trans s(s(X:Int)) => X ;)
============================ GOAL 1-1 ============================
<
mod! SPEC{
	[Zero < Int]
	op 0 : -> Zero {constr}
	ops s_ p_ : Int -> Int {constr}
	eq [eq1] : s p X:Int = X:Int .
	eq [eq2] : p s X:Int = X:Int .
	trans [tr1] : s 0 => 0 .
},
	  eq s X:Int
	    = p s s X:Int .
	  trans s s X:Int
	    => X:Int . >
unproved

INFO: an initial goal generated!
\end{verbatim}
}

Note that now the original CafeOBJ module is displayed. Moreover, the prover
introduces in some cases equations and rules to the module. These new statements
will be also displayed following CafeOBJ syntax.

\item
\verb"(discard critical pair .)", which allows to discard critical pairs in equations and transitions.
This command is dealt by using the commands \verb"(equation .)", to apply an equation,
\verb"(backward equation .)", to apply an equation from right to left, \verb"(transition .)", to
apply a transition, and \verb"(backward transition .)", to apply a transition from right to left.

\item
The rest of the rules, like the ones for applying rules or instantiating lemmas,
have been internally modified to deal with CafeOBJ statements instead of Maude ones.
However, its appearance is only modified is the command above is activated.

\item
It is worth noticing that the commands are also extended with error rules dealing
with specific errors due to the interleaving of CafeOBJ and Maude specifications.
For instance, if the CafeOBJ mode is active the user cannot introduce goals on
Maude modules:

{\codesize
\begin{verbatim}
Maude> (goal MAUDE-MODULE |- eq 0 = 0 ;)
WARNING: the selected module is not a CafeOBJ specification.
\end{verbatim}
}

\end{itemize}













